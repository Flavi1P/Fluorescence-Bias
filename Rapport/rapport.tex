\documentclass[]{report}

\usepackage[T1]{fontenc}
\usepackage[utf8]{inputenc}
\usepackage[francais]{babel}
\usepackage{layout}
\usepackage[top=2.5cm, bottom=2.5cm, left=2.5cm, right=2.5cm]{geometry}
\usepackage{setspace}
\usepackage{fontspec}
\usepackage{graphicx}
\setmainfont{Calibri}

%----------------------------------------------------------------------------------------
\begin{document}
%----------------------------------------------------------------------------------------

\begin{titlepage}
\includegraphics[width=0.3\textwidth]{logo.jpg}{UNIVERSITE SORBONNE UNIVERSITE\\Paris, FRANCE}

\end{titlepage}


\begin{onehalfspace}


\part{La fluorescence}
\chapter{Les rendements}
\section{les classes de taille}
Dans la présentation des méthodes numériques (modèles, statistiques …), vous préciserez vos choix et
développerez les méthodes de telle sorte que les calculs puissent être refaits par le lecteur à partir de
n’importe quel support informatique. Vous ne préciserez si besoin le logiciel que vous avez utilisé (titre,
version, développeur) qu’en toute fin de cette partie.
Le manuscrit doit être clair et concis, sous une forme rédactionnelle évitant d’être morcelé par des
classifications trop nombreuses (chapitre, sous chapitre, ...). La numérotation des paragraphes ne doit
pas dépasser 3 étages (par exemple 1.2.1).
Le texte devra être dactylographié en [interligne 1,5], avec des marges de [2,5 cm] en haut, bas, droite et
gauche, afin d'en rendre la lecture plus aisée.
La police de caractère de référence est « Calibri 12 » ou équivalent. La couleur de référence du texte est
le noir.
Le nombre maximum de pages est de 30, figures et tableaux inclus. Les pages doivent être numérotées
(1 à maximum 30) en bas à droite.
Des annexes peuvent être ajoutées. Elles contiennent exclusivement des compléments pouvant éclairer
sur les données ou des détails techniques de la partie « matériel et méthodes ».

\end{onehalfspace}
\end{document}